\documentclass{article}                          %%%定义文档类型
\usepackage[utf8]{inputenc}
\usepackage{ctex}                                %%%中文支持
\usepackage{amsmath}                             %%%数学公式包
\numberwithin{equation}{subsection}              %%%公式自动编号
\usepackage{geometry}
\geometry{left=0.95in,right=0.95in,top=1in,bottom=1in}
\usepackage{graphicx}                            %%%图片支持
\usepackage{caption}
\title{任意二维直线的高斯节点与高斯权重}
\author{李晓东,中国地质大学(武汉)工程学院}

\begin{document}
	\maketitle
	\begin{abstract}
	本文给出了任意二维直线的高斯节点与高斯权重系数,首先给出了水平和垂直直线上的高斯积分,而后给出了和水平方向夹角小于90度的直线上的高斯积分公式。	
	\end{abstract}

\section{水平和垂直直线上的高斯积分}
对于水平或者垂直直线上的高斯积分,直接套用一维任意曲线的高斯积分公式即可:
\begin{equation}\label{key}
\int_{a}^{b} f(t) dt= \frac{b-a}{2} \sum_{i=0}^{N} A_i f\left(\frac{b-a}{2} x_i+ \frac{a+b}{2} \right)
\end{equation}
只是在程序编写的时候,需要判定一下是水平直线还是垂直直线,而后直接按照其中一个坐标套用上式即可。

\section{和水平方向夹角小于90度的高斯积分公式}
根据微积分理论:
\begin{equation}\label{key}
I=\int_{\Gamma}f(x,y)ds=\int_{a}^{b}f(x,y)\sqrt{1+\left( \frac{dy}{dx}\right) ^2}dx
\end{equation}
因为是直线,所以记$ J=\sqrt{1+\left( \frac{dy}{dx}\right) ^2} $,$ J $是定值,所以积分公式变为:

\begin{equation}\label{key}
I=\frac{b-a}{2} J\sum_{i=0}^{N} A_i f\left(\frac{b-a}{2} x_i+ \frac{a+b}{2} ,y_i \right)
\end{equation}
其中:
\begin{equation}\label{key}
y_i=y_1+\frac{y_2-y_1}{x_2-x_1} (\frac{b-a}{2} x_i+ \frac{a+b}{2}-x_1)
\end{equation}
\end{document}